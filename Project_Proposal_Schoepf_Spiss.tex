%----------------------------------------------------------------------------------------
% Einführung in das Wissenschaftliche Arbeiten 
% Report LaTeX template (English)
% Interactive Graphics and Simulation Group
% University of Innsbruck
%----------------------------------------------------------------------------------------

\documentclass[11pt,a4paper]{article}

%----------------------------------------------------------------------------------------
% Include required packages
\usepackage{graphicx}
\usepackage{amsmath}
\usepackage{pdfpages}
\usepackage{url}
\usepackage{subfig}
\usepackage{fancyhdr}
\usepackage{lastpage}
\usepackage{float}
\usepackage[english]{babel}
\usepackage{hyperref}
%\usepackage[ngerman]{babel}

\usepackage{listings}
\usepackage{xcolor}
\lstset {
    backgroundcolor=\color{black!5}, % set backgroundcolor
    basicstyle=\scriptsize,% basic font setting
    captionpos=b,
    numbers = left,
    breaklines=true,
}

%\usepackage[utf8x]{inputenc}
%\usepackage[T1]{fontenc}

\usepackage[left=2.7cm, right=2.7cm, top=3cm]{geometry}

% Set up the header and footer
\pagestyle{fancy}
\rhead{}
\lhead{}
\chead{}
\lfoot{Sch{\"o}pf, Spiss} % Bottom left footer
\cfoot{Physical based simulation - Project Proposal} % Bottom center footer
\rfoot{Page\ \thepage\ of\ \protect\pageref{LastPage}} % Bottom right footer
\renewcommand\footrulewidth{0.4pt} % Size of the footer rule
\renewcommand{\headrulewidth}{0pt}
\setlength\parindent{0pt} % Removes all indentation from paragraphs

%----------------------------------------------------------------------------------------
% Start document

\begin{document}
\begin{center}
{\LARGE \bf Project Proposal}

Physical based simulation PS\\[5mm]

Lukas Sch\"opf\\Stefan Spiss\\[5mm]

Innsbruck, \today
\end{center}


%----------------------------------------------------------------------------------------
% Main body
%----------------------------------------------------------------------------------------

\section{Overall idea}
\label{sec:overall}
Implementation of a game similar to the game shown in the video at the following link:\\
\href{https://www.youtube.com/watch?v=JutEsb0ye94}{https://www.youtube.com/watch?v=JutEsb0ye94}\\\\
Additionally, also an example App for that game can be found at: \\
\href{http://www.mobogenie.com/download-labyrinth-ball-in-balance-2d-1746679.html}{http://www.mobogenie.com/download-labyrinth-ball-in-balance-2d-1746679.html}\\

In the game a labyrinth is shown from above and the goal is to navigate a ball from start to goal. The ball is controlled by tilting the board. 

\section{Technical implementation}
\label{sec:technical}
\begin{itemize}
\item 3D model of labyrinth
\item Rendering in OpenGL combined with SDL 2.0
\item Tilting controlled by mouse or keyboard
\item Physical-based implementation:
\begin{itemize}
\item Ball and maze are rigid bodies
\item Ball can collide with maze
\item Ball starts to roll according to gravity, when the board is tilted
\item Additionally, adding roll friction to the ball
\end{itemize}
\end{itemize}

\section{Development roadmap}
\label{sec:roadmap}
\begin{enumerate}
\item Setting up the whole graphical visualisation and the controlling interface.
\item Implementing the rigid body simulation.
\item Implementing collisions of the rigid bodies.
\item Implementing gravity-induced rolling of the ball.
\item Adding game elements (start, win, fail)
\end{enumerate}
%----------------------------------------------------------------------------------------
% Bibliography
%----------------------------------------------------------------------------------------



\end{document}
